% !TEX program = xelatex
\documentclass[12pt,a4paper]{article}

% 数学公式相关包
\usepackage{amsmath}
\usepackage{amsfonts}
\usepackage{amssymb}
\usepackage{mathrsfs}
\usepackage{bm}

% 其他常用包
\usepackage{graphicx}
\usepackage{geometry}
\usepackage{float}
\usepackage{caption}
\usepackage{subcaption}

% 页面设置
\geometry{left=2.5cm,right=2.5cm,top=2.5cm,bottom=2.5cm}

% 中文支持
\usepackage{ctex}

% 自定义命令
\newcommand{\diff}{\mathrm{d}} % 微分符号
\newcommand{\pdiff}{\partial} % 偏微分符号
\newcommand{\e}{\mathrm{e}} % 自然对数的底
\newcommand{\im}{\mathrm{i}} % 虚数单位
\newcommand{\bvec}[1]{\mathbf{#1}} % 粗体向量

\title{数学推导}
\author{火兴辉}
\date{\today}

\begin{document}
\maketitle

\section{无量纲量}

\subsection{密度比}
\begin{equation}
    s = \frac{\rho_p}{\rho_f}
\end{equation}
where $\rho_p$ is the particle density and $\rho_f$ is the fluid density.

\subsection{考虑浮力的有效重力加速度}
\begin{equation}
    \hat{g} = g(1-\frac{1}{s})
\end{equation}
where $g$ is the gravity acceleration.

$\sqrt{s}$ separates the settling velocity scale $\sqrt{s \hat{g} d}$ from the escape velocity scale $\sqrt{\hat{g} d}$ that grains need to exceed in order to escape the potential traps of the bed surface.

\subsection{Galileo Number}
\begin{equation}
    Ga = \frac{d \sqrt{s \hat{g} d}}{\nu_f}
\end{equation}
where $d$ is the particle diameter and $\nu_f$ is the kinematic viscosity.

$Ga$ controls the scaling of the dimensionless terminal grain settling velocity in quiescent flow $v_s^-/\sqrt{s \hat{g} d}$ (Camenen, 2007). 

\subsection{Shields Number}
\begin{equation}
    \Theta = \frac{\tau}{\rho_p \hat{g} d}
\end{equation}
where $\tau$ is the fluid shear stress on the bed surface.

The Shields Number $\Theta$ is a measure for the ratio of the shear stress on the bed surface to the effective weight of the sediment particle (P\"ahtz et al., 2020).

\subsection{空中颗粒承载量}

The transport load $M$ is the total mass of transported particles in the air per unit bed area. Non-dimensionalized by the particle density and diameter, it is defined as
\begin{equation}
    M_* = \frac{M}{\rho_p d}.
\end{equation}

\subsection{流向颗粒通量}
\begin{equation}
    Q = M \overline{v_x}
\end{equation}
where $\overline{v_x}$ is the streamwise average particle velocity. Non-dimensionalized as
\begin{equation}
    \overline{v_{x*}} = \frac{\overline{v_x}}{\sqrt{s \hat{g} d}}.
\end{equation}
while the non-dimensional streamwise particle transport rate is defined as
\begin{equation}
    Q_* = \frac{Q}{\rho_p d \sqrt{s \hat{g} d}}.
\end{equation}

\section{输沙率表达式}
P\"ahtz et al. (2020) proposed new expressions for $M_*$ and $Q_*$.
\begin{equation}
    M_* = (\Theta - \Theta_t)/\mu_b,
\end{equation}
\begin{equation}
    Q_* = M_* \overline{v_{x*}}_t (1 + c_M M_*) \label{eq:Qstar},
\end{equation}
where $\mu_b$ is the bed friction coefficient, which approximates the ratio between the average streamwise momentum loss and vertical momentum of transported particles during their contacts with the bed surface (P\"ahtz et al., 2018). The subscripts $t$ refers to the threshold value and $\Theta_t$ is the partical transport threshold (i.e., $M_* \rightarrow 0$ when $\Theta \rightarrow \Theta_t$).

Equation \ref{eq:Qstar} consists of two additive contributions to $Q_*$: the term $M_* \overline{v_{x*}}_t$ corresponds to the scaling of $Q_*$ in the hypothetical case that collisions between transported particles are ommitted, and the term $c_M M_* \overline{v_{x*}}_t$ accounts for such collisions.

P\"ahtz et al. (2020) found that Eq. \ref{eq:Qstar} with $c_M = 1.7$ is universally valid across a wide range of equilibrium non-suspended transport simulations ($s \in [2.65, 2000] and Ga \in [5, 100]$) that satisfy $s^{1/2} Ga \geq 80$ for $s \leq 10$ (typical for fluvial environments) or $s^{1/2} Ga \geq 200$ for $s \leq 10$ (typical for aeolian environments). The value of $\Theta_t$ is fitted to experimental and numerical data, while $\mu_b$ and $\overline{v_{x*}}_t$ are derived from semi-empirical relations: $\mu_b = 0.63$ (P\"ahtz and Dur\'an, 2018) and $\overline{v_{x*}}_t \approx 2 \kappa^{-1} \sqrt{\Theta_t}$ (limited to $s^{1/4} Ga \geq 40$, P\"ahtz and Dur\'an, 2018).

\section{颗粒跃移模型}
\subsection{流场廓线}
\begin{equation}
    u_x = \frac{u^*}{\kappa}ln\frac{z}{z_0} \label{eq:ux},
\end{equation}
where $u_x$ is the wind speed, $u^*$ is the friction velocity, $\kappa = 0.41$ is the von Karman constant, $z$ is the height, and $z_0=d_{90}/30$ is the roughness length.

We use non-dimensional arguments to rewrite Eq. \ref{eq:ux} as
\begin{equation}
    u_x = u^* f_u(z),
\end{equation}
with
\begin{equation}
    f_u(z) = \frac{1}{\kappa}ln\frac{z}{z_0}.
\end{equation}
\subsection{流场对颗粒的拖曳力}
Define the fluid velocity as $\vec{u}$ and the particle velocity as $\vec{v}$. Considering the particle as a rigid sphere, the drag force exerted by the fluid on the particle can be expressed as
\begin{equation}
    F_d = \frac{\pi}{8} \rho_f C_d d^2 |\vec{u} - \vec{v}| (\vec{u} - \vec{v}),
\end{equation}
where $C_d$ is the drag coefficient.
\begin{equation}
    C_d = \left(\sqrt{C_d^\infty} + \sqrt{\frac{R_u^c}{R_u}}\right)^2,
\end{equation}
where $R_u=|\vec{u} - \vec{v}|d/\nu_f$ is the particle Reynolds number based on the fluid-particle relative velocity, $R_u^c \approx 24$ is the critical Reynolds number above which the drag coefficient becomes almost constant, and $C_d^\infty \approx 0.5$ is the drag coefficient in the high Reynolds number limit.

In the fluid field, the grain acceleration $\vec{a}$ consists of a drag (superscript $d$), a gravity (superscript $g$), and a buoyancy (superscript $b$) component:
\begin{equation}
    \vec{a} = \vec{a^d} + \vec{a^g} + \vec{a^b}.
\end{equation}
In particular, if we only consider a 2D problem, acceleration in each direction writes
\begin{equation}
    \begin{cases}
        a_x = a_{x}^d, \\
        a_z = a_{z}^d + a_{z}^g + a_{z}^b = a_{z}^d - \hat{g}.
    \end{cases}
\end{equation}

\subsection{床面摩擦系数}
The bed friction coefficient $\mu_b$ is linked to the streamwise and vertical components of the particle acceleration due to the non-contact forces via (P\"ahtz and Dur\'an, 2018; P\"ahtz et al., 2020)
\begin{equation}
    \mu_b = -\frac{\overline{a_x}}{\overline{a_z}} = \frac{\overline{a_{x}^d}}{\overline{a_{z}^d} - \hat{g}}.
\end{equation}
 The overbar denotes the particle concentration-weighted average, which is defined as
\begin{equation}
    \overline{A} = \frac{\int_0^{z_{max}} \phi A \diff z}{\int_0^{z_{max}} \phi \diff z}.
\end{equation}
where $\phi$ is the particle volume fraction, and $z_{max}$ is the top of the transport layer. In fact, one can derive that $M = \int_0^{z_{max}} \phi \diff z$.

Now we check the expressions for $\overline{a_{x}^d}$ and $\overline{a_{z}^d}$. To simplify them, a linearization is applied by approximating the difference $|\vec{u} - \vec{v}|$ by $\overline{u_x} - \overline{v_x}$. Then the drag acceleration is given by
\begin{equation}
    \vec{a^d} = \frac{F_{dx}}{(1/6)\pi d^3 \rho_p} = \frac{3 \Delta V}{4s d} \left(\sqrt{\frac{1}{2}} + \sqrt{\frac{24 \sqrt{s \hat{g} d}}{\Delta V Ga}} \right)^2 \left(\vec{u} - \vec{v}\right),
\end{equation}
with
\begin{equation}
    \Delta V = |\overline{u_x} - \overline{v_x}|.
\end{equation}
So we have the weighted average of the x-component of the drag acceleration as
\begin{equation}
    \overline{a_{x}^d} = \frac{3 (\Delta V)^2}{4s d} \left(\sqrt{\frac{1}{2}} + \sqrt{\frac{24 \sqrt{s \hat{g} d}}{\Delta V Ga}} \right)^2 \label{eq:axd}.
\end{equation}
Since the vertical component of the fluid velocity $u_z = 0$, and $\overline{v_z} = 0$ due to the mass conservation, we have
\begin{equation}
    \overline{a_{z}^d} = \frac{3 \Delta V}{4s d} \left(\sqrt{\frac{1}{2}} + \sqrt{\frac{24 \sqrt{s \hat{g} d}}{\Delta V Ga}} \right)^2 (\overline{u_z} - \overline{v_z}) = 0.
\end{equation}

Finally, $\mu_b$ is given by
\begin{equation}
    \mu_b = \frac{\overline{a_{x}^d}}{\hat{g}} = \frac{\Delta V}{v_s} \label{eq:mub},
\end{equation}
where $v_s$ is the particle velocity when $a_z^d = \hat{g}$. In other words, it is the terminal settling velocity.

$V_s$ is the dimensionless form of $v_s$, which is useful in subsequent derivations. As $v_s = \Delta V / \mu_b$, $V_s$ can be written as
\begin{equation}
    V_s = \frac{v_s}{\sqrt{s \hat{g} d}} = \frac{\Delta V}{\mu_b \sqrt{s \hat{g} d}}.
\end{equation}
Substituting Eq. \ref{eq:mub} into Eq. \ref{eq:axd}, we have a quadratic equation for $\sqrt{\Delta V}$:
\begin{equation}
    \Delta V \sqrt{\frac{1}{2}} + \sqrt{\frac{24 \Delta V \sqrt{s \hat{g} d}}{Ga}} = \sqrt{\frac{4}{3} \mu_b s \hat{g} d}.
\end{equation}
With the solution of $\Delta V$, $V_s$ can be derived as
\begin{equation}
    {V_s = \frac{1}{2\mu_b} \left( -\sqrt{\frac{24}{Ga}} + \sqrt{\frac{24}{Ga} + 4\sqrt{\frac{2\mu_b}{3}}} \right)^2 \label{eq:Vs}}.
\end{equation}
Note that $v_s^-$ is distinct from $v_s$, which obeys a modified version of Eq. \ref{eq:Vs} with $\mu_b = 1$ (Camenen, 2007).

% 示例:多行公式对齐
\begin{align}
    \frac{\pdiff f}{\pdiff t} &= \frac{\pdiff}{\pdiff t}\int_a^b g(x,t)\diff x \\
    &= \int_a^b \frac{\pdiff g}{\pdiff t}\diff x
\end{align}

% 示例:分段函数
\[
    f(x) = \begin{cases}
        x^2, & x \geq 0 \\
        -x^2, & x < 0
    \end{cases}
\]

% 示例:矩阵
\[
    \begin{bmatrix}
        a_{11} & a_{12} \\
        a_{21} & a_{22}
    \end{bmatrix}
    \begin{bmatrix}
        x_1 \\
        x_2
    \end{bmatrix} =
    \begin{bmatrix}
        b_1 \\
        b_2
    \end{bmatrix}
\]

\section{推导过程}
% 这里可以继续您的推导...

\end{document} 